% Classe de document.
\documentclass[a4paper]{article}


%%% ENCODAGE ET LANGUES %%%
% Encodage de l'entrée.
\usepackage[utf8]{inputenc}

% Encodage de la police.
\usepackage[T1]{fontenc}

% Langues à charger.
\usepackage[francais]{babel}
\frenchbsetup{og=«,fg=»}

\newcommand\ttt\texttt

%%% AUTRES PAQUETS %%%
\usepackage{hyperref}

\title{Projet Programmation 2, partie 2}
\author{Glen \textsc{Mével} \& Guillaume \textsc{Aubian}}
\date{8 avril 2015}

\begin{document}

\maketitle

	\begin{abstract}
	Voici le rapport de notre projet Programmation 2 trouvable ici:
	\url{https://github.com/gmevel/ScalaPuzzles}.
	\end{abstract}

\section*{Introduction}

	\paragraph{}
	Les trois jeux que nous avons choisis sont \textsl{Flip}, \textsl{Unruly} et
	\textsl{Flood}. Le jeu \textsl{Minesweeper} n'a pas été modifié sauf pour
	re-factoriser le code de l’interface graphique.

	\section{Interface}

	\paragraph{}
	Le degré de réutilisabilité que nous avions obtenu lors de la partie 1 était
	très insuffisant pour la suite (seul la grille de base était générique).
	Nous avons donc largement approfondi ce point. Le code des interfaces
	graphiques des différents jeux est maintenant très factorisé et il ne reste
	que les portions complètement spécifiques à chaque jeu.

	\paragraph{}
	Pour chaque jeu, une partie est implémentée par une classe \ttt{Board} qui
	suit le trait \ttt{Game.Board}. Elle ne gère que le cœur du jeu sans aucun
	affichage, qui est effectué par les classes suivantes.

	\subsection{La classe \ttt{Game.BoardUI} pour une partie de jeu}

	\paragraph{}
	Nous avons repris la classe générique \ttt{GridUI} qui affiche une grille,
	mais celle-ci n’est plus utilisée directement par les jeux. Nous avons
	introduit la classe abstraite \ttt{Game.BoardUI} pour implémenter une
	interface générique d’une partie de jeu (donnée en paramètre)~; elle
	contient une grille de type \ttt{Util.GridUI}, ainsi qu’une barre de statut
	avec le bouton-smiley, le bouton « restart » et le chronomètre. Cette classe
	abstraite gère entre autres l’action du bouton-smiley en fonction de l’état
	du jeu. On l’étend pour définir les actions possibles et le traitement des
	événements spécifiques à chaque jeu.

	\subsection{La classe \ttt{Game.UI} pour une interface de jeu complète}

	\paragraph{}
	Nous avons ensuite une classe abstraite \ttt{Game.UI} qui crée l’interface
	complète d’un jeu, avec les menus pour lancer une nouvelle partie, la
	fenêtre de configuration personnalisée, la fenêtre d’aide et la partie
	actuelle (un composant Swing de type \ttt{Game.BoardUI}).

	\paragraph{}
	On étend la classe \ttt{Game.UI} en renseignant les méthodes et valeurs
	utiles pour l’interface d’un jeu (cette classe impose implicitement une
	structure de code commune à tous les jeux). Entre autres, on précise quels
	sont les paramêtres d’une partie (largeur, hauteur, etc.), et les
	éventuelles contraintes associées (par exemple, il ne doit pas y avoir dans
	une partie de démineur plus de mines que de cases dans la grille). Ces
	données permettent à la classe \ttt{Game.UI} de proposer différents modes de
	jeu prédéfinis et de générer automatiquement la fenêtre de partie
	personnalisée.

	\paragraph{}
	Initialement, nous avions cherché à réaliser un système de paramètres
	pouvant être de différents types~: entiers, booléens, chaînes de caractères
	ou variantes. C’était fastidieux, et la méthode la plus économique
	nécessitait des conversions explicites de type. Finalement, il est apparu
	que nous n’avions besoin que de paramètres entiers (hormis la graine
	alphanumérique, mais elle est commune à tous les jeux et nous ne l’incluons
	pas dans les paramètres), et nous avons donc renoncé à ce système dans toute
	sa généralité.

	\subsection{La fenêtre principale}

	\paragraph{}
	Enfin, l’interface racine est implémentée par l’objet \ttt{Main}. Il affiche
	un écran d’accueil et permet de choisir entre les différents jeux.
	L’interface est implémentée sous forme d’onglets, un par jeu. Sélectionner
	l’onglet d’un jeu lance automatiquement une partie. Afin que la fenêtre
	prenne la taille la plus adaptée à l’onglet actuel sans tenir compte de la
	taille des onglets inactifs, nous avons dû définir notre propre composant
	Swing (\ttt{Util.ResponsiveTabbedPane}) en étendant le composant existant
	(\ttt{TabbedPane}).

	\subsection{Structure du code}

	\paragraph{}
	Le code de chaque jeu est divisé comme suit~:
	\begin{itemize}
		\item un fichier \ttt{Board.scala} implémentant le cœur du jeu (la
			classe \ttt{Board}),
		\item un fichier \ttt{BoardUI.scala} implémentant l’interface graphique
			d’une partie (la classe \ttt{BoardUI}),
		\item et un fichier \ttt{UI.scala} implémentant l’interface complète du
			jeu (l’objet \ttt{UI}).
	\end{itemize}
	Le tout est regroupé dans un paquet portant le nom du jeu. Le paquet
	\ttt{Game}, définissant les classes génériques correspondantes, suit la même
	structure.

	\paragraph{}
	Les classes et méthodes utiles mais sans rapport direct avec le projet sont
	rangées dans un paquet \ttt{Util}.

\section{Jeux}

	\paragraph{}
	Dans chacun des jeux que nous avons codés, nous avons implémenté la
	possibilité de recommencer la partie, ce que nous faisons à chaque fois en
	recopiant la matrice qui représente la grille, et en stockant de coté toutes
	les variables du jeu en question.

	\subsection{\textsl{Flip}}

	\paragraph{}
	Nous n'avons pas rencontré de difficulté particulière pour l'implémentation
	du jeu \textsl{Flip}. La seule astuce que nous avons utilisée consiste à
	compter le nombre de cases qu'il reste à retourner, en mettant à jour ce
	compteur lors de chaque mouvement. Ainsi, tester si la partie est terminée
	revient à regarder si ce compteur est nul.

	\paragraph{}
	Nous proposons aussi la possibilité de revenir en arrière (avec le bouton
	smiley). L’historique s'implémente aisément avec une pile.

	\subsection{\textsl{Unruly}}

	\paragraph{}
	La principale difficulté de \textsl{Unruly} est de générer les grilles de
	manière à s'assurer qu'il existe au moins une solution. Pour cela, notre
	stratégie consiste à créer une grille solution, puis à enlever des cases au
	hasard dans cette grille (une proportion de cases révélées de $1/3$ nous a
	semblé satisfaisante).

	\paragraph{}
	Pour créer notre grille solution, nous avons procédé par backtracking~:
	étant donnée une case, on la colorie aléatoirement, et on essaie de remplir
	le reste de la grille par récurrence. Si ce n’est pas possible, on réessaie
	en coloriant la-dite case de l'autre couleur. Si ce n’est toujours pas
	possible, c'est qu'il n'y a plus de grille possible.

	\paragraph{}
	On doit vérifier après chaque coup sa validité et si le jeu est terminé.
	Pour le second test, une technique efficace consiste à mémoriser le nombre
	de cases vides dans une variable nommée \ttt{voidC}, que l'on met à jour
	à chaque coup. Ainsi, la condition de victoire s’écrit simplement
	\ttt{voidC == 0}. Pour s’assurer que la partie se termine effectivement
	lorsque toutes les cases sont coloriées, on n’autorise que les coups
	valides.

	\paragraph{}
	Nous avons implémenté la possibilité d'abandonner la partie en cours, et de
	voir la solution, avec le bouton smiley.

	\subsection{\textsl{Flood}}

	\paragraph{}
	Il s'agit du jeu le plus difficile à coder des trois. En plus de leur
	couleur, on mémorise pour chaque cellule si elle appartient à la zone
	« contrôlée ». Par ailleurs, on maintient une liste des cases contrôlées et
	une pile des cases qui sont à la frontière de la zone controlée.

	\paragraph{}
	La mise à jour de la zone contrôlée et de sa frontière se fait grâce à la
	fonction \ttt{expandCell} qui, étant donnée une case, la marque comme
	contrôlée, l’ajoute (ou la laisse) dans la frontière si elle a au moins une
	voisine de couleur différente, puis s'applique à toutes ses voisines qui de
	même couleur et non contrôlées. Étendre la zone revient donc à appliquer
	\ttt{expandCell} sur toutes les cases à la frontière.

\end{document}
